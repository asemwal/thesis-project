\section{Introduction}
\label{sec:intro}

Monitoring BGP\cite{bgp} routing is important for both operations and research.   To provide access to BGP data,  a number of public and private BGP monitors are deployed and widely used \cite{routeviews,riperis}.    BGPmon is part of the Oregon RouteViews project\cite{routeviews} and uses a publish/subscribe overlay network to provide real-time access to vast numbers of peers and clients.   For over decade,  RouteViews has collected BGP routing updates and BGP routing tables from routers around the globe.    This data, in MRT format\cite{mrt},  is publicly available from \emph{http://archive.routeviews.org}.   The archives include historical data dating back a decade as well as relatively recent data (within the last few minutes/hours).      BGPmon builds on this successful system and extends this infrastructure in three ways.   

First,  BGPmon provides a real-time feed of both BGP updates and routing tables.    Instead of downloading data from a site (which could incur delays of hours),   a user can simply open a TCP connection and receive that data in real-time (which incurs a delay on the order of network propagation times).     This change to easily accessible real-time data opens up a range of new opportunities for tools and live analysis.    

Second, BGPmon provides the data in an XML format.   The format includes both binary ``bits off the wire'' attributes and more human/parser friendly ASCII text in the same message format.    Since the format is XML,  users can easily add new tags while still maintaining a consistent format that can be shared between sites.  A site might tag particular messages with a local attribute (such as my prefix or some other label meaningful only to that site).    Other sites can still use the data,  either stripping the tag before exchanging data if that tag has private information or simply sending the data with the new tag as other XML parsers can simply ignore unrecognized tags.      Similarly,   a site can easily drop some tags from the format.  

For example,   BGP messages must include a timestamp.   However different applications may desire different types of time representations.   In some cases, a unix timestamp is the preferred representation and the MRT format uses a unix timestamp.    But while timestamps are convenient for coding,  other applications may want a finer granularity of milliseconds.   In yet another example, if the data is to be viewed by humans rather than processed by code,   a human friendly representation listing year, month, day, and time is preferred.    The XML format supports all three ways of displaying time.     In fact, the XML stream includes all three time representations (using different XML attributes in the $<$TIME$>$ section of the message).    Applications desiring one time format can simply use that attribute and ignore the others.   Applications can also strip out the other time representations to save space and later other users can reconstruct the human friendly time from the timestamp (or vice versa).

Third,  for users gathering their own data or aggregating streams of data,  BGPmon provides a new approach to data collection.  Existing monitors typically collect data using a full implementation of a BGP router.   In contrast, BGPmon eliminates the unnecessary functions of route selection and data forwarding to focus only on the monitoring function.  In its place, BGPmon adds chaining functions so that feeds can better scale and can be combined to a large-scale mesh monitoring infrastructure.    For example,   one can take the real-time feed from RouteViews, chain it with feeds from local BGPmon,  and direct the results into a few servers that provide the organization with real-time BGP access.