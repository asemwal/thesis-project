\section{Login Module}
\label{sec:login}

The login module handles the Cisco-like commands typed by logined users and calls the corresponding functions provided by other modules. It is made up of several pieces.  

The first piece is a listener that listens on a specific address and port for new Command Line Interface (CLI) connections.  Once a connection has been established then the next major piece, the command tree structure, is used.  This structure is a tree ADT that contains a complete mapping of all the commands that can be called from the CLI.  Each command is broken down into parts and each part is added to the tree.  For instance the command 'show running' is broken into two pieces.  The 'show' command is a child of the root node and the 'running' command is child of the 'show' command.  When a command is executed from the CLI, the command is sent from the client to the server then broken down and mapped into the command structure.  If the entire command can be mapped onto the tree then the final node will contain a function pointer for the command.  Every node in the structure also has an associated mask with it that controls whether the command is visible or executable to a given security level.  All the commands have been organized in a way that will help make maintenance of them easier.  In commandprompt.c there are a series of functions that contain the necessary code to create the commands and associations to the necessary functions.  Then, there is a series of files that end with '\_commands.c' which contain the functions for each module that the commands reference.